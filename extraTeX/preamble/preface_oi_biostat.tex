

\chapter*{Preface}


This book introduces statistics and its applications in the life sciences and biomedical research.  It is based on the freely available \textsl{OpenIntro Statistics, Third Edition}, and, like \textsl{OpenIntro}, it may be downloaded as a free PDF at \url{https://github.com/OI-Biostat/oi_biostat_text}.  The text adds substantial new material, eliminates sections from \textsl{OpenIntro} that are less relevant to the life sciences, and re-uses some \textsl{OpenIntro} material directly. We have retained some of the examples and exercises from \textsl{OpenIntro} that may not come directly from medicine or the life sciences but illustrate important ideas or methods. 

\textsl{Introduction to Statistics for the Life and Biomedical Sciences} is intended for undergraduate and graduate students interested in careers in biology or medicine, and may also be profitably read by students of public health or medicine.  It covers many of the traditional introductory topics in statistics, in addition to discussing some newer methods being used in molecular biology. 

Statistics has become an integral part of research in medicine and biology, and the tools for summarizing data and drawing inferences from data are essential both for understanding the outcomes of studies and for incorporating measures of uncertainty into that understanding.  An introductory text in statistics for students who will work in medicine, public health, or the life sciences should be more than simply the usual introduction, supplemented with an occasional example from biology or medical science. Many of the examples and exercises in this text use published data that we hope convey the value of statistics in medical and biological research. In cases where examples draw on important material in biology or medicine, the problem statement contains the necessary background information. 

Computing is an essential part of the practice of statistics.  Nearly everyone entering the biomedical sciences will need to read and interpret the results of analyses conducted in software; many will also need to be capable of directly conducting such analyses. This set of materials separates those two activities to allow students and instructors to emphasize either or both skills. The text discusses the important features of figures and tables used to support an interpretation, rather than the process of generating such material from data. This allows students whose main focus is on statistical concepts and their application to not be distracted by the details of a particular software package. In our experience, however, we have found that many students enter a research setting after only a single course in statistics. These students benefit from a practical introduction to data analysis that incorporates the use of a statistical computing language. The self-paced learning labs that can be downloaded from \url{https://github.com/OI-Biostat/oi_biostat_labs} are designed to  provide guidance on conducting data analysis and visualization with the \textsl{R} statistical language, while building understanding of statistical concepts. The labs begin from first principles and require no previous experience with statistical software. The labs in Unit 0 (\texttt{00\_getting\_started}) provide information on downloading and installing both of these freely available packages. Information on downloading and installing the packages may also be found at \href{http://www.openintro.org}{\color{black}\textbf{openintro.org}}. 

The datasets used in this book are available via the \textsf{R} \texttt{openintro} package available on CRAN\footnote{Diez DM, Barr CD, \c{C}etinkaya-Rundel M. 2012. \texttt{openintro}: OpenIntro data sets and supplement functions. \urlwofont{http://cran.r-project.org/web/packages/openintro}.}  and the \textsf{R} \texttt{oibiostat} package available via \href{<https://github.com/OI-Biostat/oi_biostat_data>}{GitHub}.

\subsection*{Textbook overview}

The chapters of this book are as follows:

\begin{description}
\setlength{\itemsep}{0mm}

\item[1. Introduction to data.] Data structures, basic data collection principles, numerical and graphical summaries, and exploratory data analysis.
\item[2. Probability.] The basic principles of probability.
\item[3. Distributions of random variables.] Introduction to random variables and distributions of discrete and continuous distributions.
\item[4. Foundations for inference.] General ideas for statistical inference in the context of estimating a population mean.
\item[5. Inference for numerical data.] Inference for one-sample and two-sample means with the $t$ distribution, power calculations for a difference of means, and ANOVA.
\item[6. Simple linear regression.] An introduction to linear regression with a single explanatory variable, evaluating model assumptions, and inference in a regression context.
\item[7. Multiple linear regression.] General multiple regression model, categorical predictors with more than two values, interaction, and model selection.
\item[8. Inference for categorical data.] Inference for proportions using the normal and chi-square distributions, as well as simulation and randomization techniques.

\end{description}

\subsection*{Examples, exercises, and appendices}

Just as in \textsl{OpenIntro Statistics, Third Edition}, examples and within-chapter exercises throughout the textbook may be identified by their distinctive bullets:

\begin{example}{Large filled bullets signal the start of an example.}
Full solutions to examples are provided within the main text and often include an accompanying table or figure.
 \end{example}

\begin{exercise}
Empty bullets signal readers that an exercise has been inserted into the text for additional practice and guidance. Solutions are provided for all within-chapter exercises in footnotes.\footnote{Full solutions are located in the footnotes.}
\end{exercise}

There are exercises at the end of each chapter that are useful for practice or homework assignments. Solutions are in Appendix~\ref{eoceSolutions}. 

Probability tables for the normal, $t$, and chi-square distributions are in Appendix~\ref{distributionTables}, and PDF copies of these tables are also available from \href{http://www.openintro.org}{\color{black}\textbf{openintro.org}} for anyone to download, print, share, or modify.  The labs and the text also illustrate the use of simple \textsl{R} commands to calculate probabilities from common distributions.

\subsection*{OpenIntro, online resources, and getting involved}

OpenIntro is an organization focused on developing free and affordable education materials. The first project, \emph{OpenIntro Statistics}, is intended for introductory statistics courses at the high school through university levels. Other projects examine the use of randomization methods for learning about statistics and conducting analyses (\emph{Introductory Statistics with Randomization and Simulation}) and advanced statistics that may be taught at the high school level (\emph{Advanced High School Statistics}).

We encourage anyone learning or teaching statistics to visit \textbf{openintro.org} and get involved by using the many online resources, which are all free, or by creating new material. Students can test their knowledge with practice quizzes, or try an application of concepts learned in each chapter using real data and the top-rated and free statistical software \textsl{R}. Teachers can download the source for course materials, labs, slides, data sets, \textsl{R} figures, or create their own custom quizzes and problem sets for students to take on the website. Everyone is also welcome to download this textbook as a PDF or the book's source files to create a custom version of this textbook or to simply share a copy with a friend or on a website. All of these products are free, and anyone is welcome to use these online tools and resources with or without this textbook as a companion.


\subsection*{Acknowledgements}

The \emph{OpenIntro} project would not have been possible without the dedication and volunteer hours of all those involved.  The authors of \textsl{OpenIntro Statistics} would like to thank Andrew Bray, Meenal Patel, Yongtao Guan, Filipp Brunshteyn, Rob Gould, and Chris Pope for their involvement and contributions.  
Dalene Stangl, Dave Harrington, Jan de Leeuw, Kevin Rader, and Philippe Rigollet provided valuable feedback on early editions of the text.

This text has benefited from feedback from Andrea Foulkes, Raji Balasubramanian, Curry Hilton and Kevin Rader.


